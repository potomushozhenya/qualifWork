% ОБЯЗАТЕЛЬНО ИМЕННО ТАКОЙ documentclass!
% (Основной кегль = 14pt, поэтому необходим extsizes)
% Формат, разумеется, А4
% article потому что стандарт не подразумевает разделов
% Глава = section, Параграф = subsection
% (понятия "глава" и "параграф" из документа, описывающего диплом)
\documentclass[a4paper,article,14pt]{extarticle}

% Подключаем главный пакет со всем необходимым
\usepackage{spbudiploma_tempora}

% Пакеты по желанию (самые распространенные)
% Хитрые мат. символы
\usepackage{euscript}
% Таблицы
\usepackage{longtable}
\usepackage{makecell}
% Картинки (можно встявлять даже pdf)
\usepackage[pdftex]{graphicx}
% Удобные дифференциалы
\usepackage{physics}
% Матрицы и таблицы
\usepackage{amsmath}
\usepackage{multirow}

\usepackage{amsthm,amssymb, amsmath}
\usepackage{textcomp}


\begin{document}

% Титульник в файле titlepage.tex
% --------------------- Титульник ВКР СПбГУ -----------------------------
% Автор: Тоскин Николай, itonik@me.com
% Если заметили ошибку, напишите на email
% Если хотите добавить изменение самостоятельно:
% https://github.com/itonik/spbu_diploma/
% Использованы материалы:
% habr.com/ru/post/144648/
% cpsconf.ru
% Документы ниже могут уже быть неактуальны, тем не менее за годы ничего
% нового не появилось
% Текст:
% http://edu.spbu.ru/images/data/normativ_acts/local/20181030_10432_1.pdf
% Титульный лист:
% http://edu.spbu.ru/images/data/normativ_acts/local/20180703_6616_1.pdf
% -----------------------------------------------------------------------

% Титульный лист диплома СПбГУ
% Временное удаление foot на titlepage
\newgeometry{left=30mm, top=20mm, right=15mm, bottom=20mm, nohead, nofoot}
\begin{titlepage}
\begin{center}

\textbf{Санкт--Петербургский}
\textbf{государственный университет}

\vspace{35mm}

\textbf{\textit{\large Жестоканов Евгений Вячеславович}} \\[8mm]
% Название
\textbf{\large Выпускная квалификационная работа}\\[3mm]
\textbf{\textit{\large Структурный подход в мононеявных методах Рунге — Кутты}}

\vspace{20mm}
Уровень образования: бакалавриат\\
Направление 01.03.02 «Прикладная математика и информатика»\\
Основная образовательная программа СВ.5005.2015
«Прикладная математика, фундаментальная информатика и программирование»\\
Профиль «Исследование и проектирование систем управления\\ и обработки сигналов»\\[25mm]


% Научный руководитель, рецензент
\begin{flushright}
\begin{minipage}[t]{0.65\textwidth}
{Научный руководитель:} \\
кандидат физ.-мат. наук, доцент кафедры информационных систем Еремин~Алексей Сергеевич 

\vspace{10mm}

{Рецензент:} \\
профессор, кафедра компьютерных технологий \\и систем, д.ф. - м.н. Веремей~Евгений Игоревич
\end{minipage}
\end{flushright}

\vfill 

{Санкт-Петербург}
\par{\the\year{} г.}
\end{center}
\end{titlepage}
% Возвращаем настройки geometry обратно (то, что объявлено в преамбуле)
\restoregeometry
% Добавляем 1 к счетчику страниц ПОСЛЕ titlepage, чтобы исключить 
% влияние titlepage environment
\addtocounter{page}{1}


% Содержание
\tableofcontents
\pagebreak

\specialsection{Введение}
В процессе развития классических методов численного решения обыкновенных дифференциальных уравнений (ОДУ), таких как методы Рунге — Кутты (РК), экстраполяции и Адамса, которые изначально были разработаны для ручного расчёта, наблюдается постоянное расширение спектра решаемых задач, обусловленное технологическим прогрессом в области компьютерных вычислений.

С ростом мощности вычислительных машин появляются новые возможности для решения более сложных задач. Вместе с этим возникают и новые проблемы, связанные с приближением и устойчивостью более эффективных и надёжных алгоритмов численного интегрирования систем обыкновенных дифференциальных уравнений (СОДУ).

Отсутствие однородности в реальных задачах, когда встречаются задачи, которые не являются исключительно жёсткими или нежёсткими, а представляют собой их сочетание, стало важным стимулом для улучшения и развития численных методов решения как жёстких, так и нежёстких задач.

Рассматриваемая далее система вида (\ref{eq:base_system}) возникает при описании задач небесной механики, оптимального управления, физики высоких энергий \cite{srk}.
\pagebreak

\specialsection{Постановка задачи}
Методы решения структурно разделенных систем обыкновенных дифференциальных уравнений класса A(2) хорошо себя зарекомендовали с точки зрения критерия "вычислительные затраты/точность", а также показывают себя лучше явных методов РК в численной устойчивости.

С другой стороны для жестких задач используются неявные методы РК, однако требуют наибольшего времени вычисления и ресурсов. Поскольку при их использовании возникает необходимость в решении векторной нелинейной системы.

Также для жестких задач используют моно-неявные методы РК, поскольку они хорошо себя показывают в решении жестких задач с вычислительной точки зрения \cite{mirk}.

В связи с двумя вышеперечисленными пунктами возникла мысль об объединении этих двух методов для анализа поведения полученного метода на жестких и нежестких задачах. Поскольку он должен вычисляться быстрее чем неявный РК, поскольку ему не нужно решать нелинейную систему, а только несколько нелинейных уравнений. А также должен неплохо себя показывать на жестких задачах.

В рамках данной работы планируется построить метод 4-го порядка: вывести условия порядка, предложить схему, а также рассмотреть вопрос об устойчивости данного метода. 
\pagebreak

\specialsection{Обзор литературы}
Для написания данной работы были изучены и использованы научная и учебно-методическая литература, статьи студентов факультета прикладной математики - процессов управления.

Основными источниками стали монография доктора физико-
математических наук Олемского Игоря Владимировича \cite{srk}, описывающая структурные методы и приводящая их расчетные схемы и условия порядка, и статья \cite{mirk}, отражающая моно-неявные методы и приводящяя их расчетные схемы. Информация по устойчивости была взята из магистерской диссертации Винничек Никиты Николаевича.
\pagebreak

\section{Существующие методы}
\subsection{Система}
Все дальнейшие рассуждения, кроме \ref{subsecMIRK} будем вести о следующей системе:
\begin{equation}
\begin{cases}
         \dv{y_1}{x} &= f_1(x,y_2) \\
         \dv{y_2}{x} &= f_2(x,y_1)
\end{cases}
\label{eq:base_system}
\end{equation}
\begin{equation}
    y_{s}(X_{0}) = y_{s0}, \quad s = 1,2, \quad x \in [X_{0},X_{1}] \subset R,
\end{equation}
\subsection{Структурный метод}
По \cite{srk} схема рассчета для структурного метода РК описывается:
\begin{equation}
    y_{s,i+1} = y_{s,i} + h \sum_{l=1}^{m_{s}} b_{sl} k_{sl} \quad s = 1, 2 
\end{equation}
\begin{equation}
\begin{aligned}
    k_{1l} &= 
\begin{cases}
    f_{1}(x + c_{11} h, y_{2i}), \quad l = 1 \\
    f_{1}(x + c_{1l} h, y_{2i} + h \sum_{n=1}^{l-1} a_{1ln} k_{2n}), \quad l = 2, \dotsc, m_{1}
\end{cases}
    \\k_{2l} &= 
    f_{2}(x + c_{2l} h, y_{1i} + h \sum_{n=1}^{l-1} a_{2ln} k_{1n}), \quad c_{21} \not= 0, l = 1, \dotsc, m_{2}
\end{aligned}
\end{equation}
\subsection{Мононеявный метод}\label{subsecMIRK}
По \cite{mirk} схема рассчета для моно-неявного метода РК описывается:
\begin{equation}
    y_{i+1} = y_{i} + h \sum_{l=1}^{s} b_{l} k_{l}
\end{equation}
\begin{equation}
    k_{l} = 
    f(x + c_{l} h, (1 - v_{l}) y_{1i} + v_{l} y_{1,i+1} + h \sum_{n=1}^{s} x_{ln} k_{n}), \quad l = 1, \dotsc, s
\end{equation}
\pagebreak

\section{Предложенный метод}
\subsection{Схема}
\begin{equation}
    y_{s,i+1} = y_{s,i} + h \sum_{l=1}^{m_{s}} b_{sl} k_{sl} \quad s = 1, 2
    \label{eq:method}
\end{equation}
\begin{equation}
\begin{aligned}
    k_{1l} &= 
\begin{cases}
    f_{1}(x + c_{11} h, (1 - v_{11}) y_{2i} + v_{11} y_{2,i+1}), \quad l = 1 \\
    f_{1}(x + c_{1l} h, (1 - v_{11}) y_{2i} + v_{11} y_{2,i+1} + h \sum_{n=1}^{l-1} x_{1ln} k_{2n}), \quad l = 2, \dotsc, m_{1}
\end{cases}
    \\k_{2l} &= 
    f_{2}(x + c_{2l} h, (1 - v_{2l}) y_{1i} + v_{2l} y_{1,i+1} + h \sum_{n=1}^{l-1} x_{2ln} k_{1n}), \quad c_{21} \not= 0, l = 1, \dotsc, m_{2}
\end{aligned}
\end{equation}
\begin{equation}
    s_{1} = 3 \quad s_{2} = 2
\end{equation}
\begin{equation}
I_{i} = \underbrace{\left[
    1, \dotsc, 1
    \right]}_{i}
\end{equation}
Коэффициенты в общем виде выглядят следующим образом
\begin{equation}
C_{1} =
\begin{pmatrix}
    c_{1,1} & 0 & 0 \\
    0 & c_{1,2} & 0 \\
    0 & 0 & c_{1,3}
\end{pmatrix}
\end{equation}
\begin{equation}
C_{2} =
\begin{pmatrix}
    c_{2,1} & 0 \\
    0 & c_{2,2}
\end{pmatrix}
\end{equation}
\begin{equation}
X_{1} = 
\begin{pmatrix}
    0 & 0 \\
    x_{1,2,2,1} & 0 \\
    x_{1,3,2,1} & x_{1,3,2,2}
\end{pmatrix}
\end{equation}
\begin{equation}
X_{2} = 
\begin{pmatrix}
    x_{2,1,1,1} & 0 & 0\\
    x_{2,2,1,1} & x_{2,2,1,2} & 0 
\end{pmatrix}
\end{equation}
\begin{equation}
    v_{1} = \left[
    v_{1,2,1}, v_{1,2,2}, v_{1,2,3}
    \right]
\end{equation}
\begin{equation}
    v_{2} = \left[
    v_{2,1,1}, v_{2,1,2}
    \right]
\end{equation}
\begin{equation}
    b_{1} = \left[
    b_{1,1}, b_{1,2}, b_{1,3}
    \right]
\end{equation}
\begin{equation}
    b_{2} = \left[
    b_{2,1}, b_{2,2}
    \right]
\end{equation}
Вспомогательные матрицы для удобства записи условий порядка.
\begin{equation}
A_{1} = 
\begin{pmatrix}
    v_{1,2,1} b_{2,1} & v_{1,2,1} b_{2,2} \\
    v_{1,2,2} b_{2,1} + x_{1,2,2,1} & v_{1,2,2} b_{2,2} \\
    v_{1,2,3} b_{2,1} + x_{1,3,2,1} & v_{1,2,3} b_{2,2} + x_{1,3,2,2}
\end{pmatrix}
\end{equation}
\begin{equation}
A_{2} = 
\begin{pmatrix}
    v_{2,1,1} b_{1,1} + x_{2,1,1,1} & v_{2,1,1} b_{1,2} & v_{2,1,1} b_{1,3}\\
    v_{2,1,2} b_{1,1} + x_{2,2,1,1} & v_{2,1,2} b_{1,2} + x_{2,2,1,2} & v_{2,1,2} b_{1,3}
\end{pmatrix}
\end{equation}

\subsection{Условия порядка}\label{order_conditions}
Условия порядка для предложенного метода 4-го порядка. Полученны как удовлетворенные условия на соответствующие деревья из \cite{tree}.
Далее для s = 1, 2
\begin{equation}
U_{1} = I_{3}, \quad U_{2} = I_{2}
\end{equation}
\begin{equation}
A_{s} U_{s} - C_{s} U_{s}
\end{equation}
\begin{equation}
B_{s} U_{s} = 1
\end{equation}
\begin{equation}
B_{s} C_{s}^{i} U_{s} = \frac{1}{i+1}, \quad i = 1, 2, 3
\end{equation}
\begin{equation}
B_{s} A_{s} C_{s}^{i} U_{s} = \frac{1}{6*i}, \quad i = 1, 2
\end{equation}
\begin{equation}
B_{s} C_{s} A_{s} C_{s} U_{s} = \frac{1}{8}
\end{equation}
\begin{equation}
B_{s} A_{s} A_{s} C_{s} U_{s} = \frac{1}{24}
\end{equation}

\subsection{Удовлетворенные условия}\label{conditions}
Далее будут приведены коэффициенты, удовлетворяющие условиям из \ref{order_conditions}. При их вычислении два раза были выбраны корни для квадратных уравнений, после чего получилась система с одним параметром. Параметр был выбран равный 0.

\begin{equation}
    c_{1} = \left[
    1, \frac{2}{3} + \frac{\sqrt{2}}{6}, \frac{\sqrt{2}}{6}
    \right]
\end{equation}
\begin{equation}
    c_{2} = \left[
    \frac{1}{2} - \frac{\sqrt{3}}{6}, \frac{2}{3} + \frac{\sqrt{2}}{6}, \frac{\sqrt{2}}{6}
    \right]
\end{equation}
\begin{equation}
    v_{1} = \left[
    1, \frac{2}{3} + \frac{\sqrt{2}}{6} - \frac{\sqrt{3}}{6} + \frac{\sqrt{6}}{18}, \frac{-\sqrt{6}}{6} + \frac{\sqrt{2}}{6} + \frac{\sqrt{3}}{18}
    \right]
\end{equation}
\begin{equation}
    v_{2} = \left[
    \frac{2}{3} - \frac{\sqrt{3}}{6}, \frac{4}{3} - \frac{\sqrt{2}}{3} + \frac{\sqrt{3}}{6}
    \right]
\end{equation}
\begin{equation}
    b_{1} = \left[
    \frac{-1}{17} - \frac{3\sqrt{2}}{17}, \frac{3}{4}, \frac{21}{68} + \frac{3\sqrt{2}}{17}
    \right]
\end{equation}
\begin{equation}
    b_{2} = \left[
    \frac{1}{2}, \frac{1}{2}
    \right]
\end{equation}
\begin{equation}
X1 = 
\begin{pmatrix}
    0, 0\\
    \frac{-\sqrt{6}}{18} + \frac{\sqrt{3}}{6}, 0\\
    \frac{\sqrt{6}}{6} - \frac{\sqrt{3}}{18}, 0
\end{pmatrix}
\end{equation}
\begin{equation}
X2 = 
\begin{pmatrix}
    \frac{-1}{6}, 0, 0\\
    \frac{1}{6} + \frac{\sqrt{2}}{3}, -1, 0
\end{pmatrix}
\end{equation}

Удовлетворенные условия порядка в общем виде можно посмотреть в репозитории \cite{code}.
\pagebreak

\section{Численные эксперименты}
Будем проводить эксперименты для установления порядка метода. Для эксеприментов будем использовать мною написанный код с вышепредложенным методом \cite{code}. \newline
Рассмотрим задачу вида
\begin{equation}
\begin{aligned}
    y_{1}' &= -y_{2} + e^{-x} \\
    y_{2}' &= y_{1} + e^{-x} \\
    x_{0} &= 0, y_{1}(0) = y_{2}(0) = 1, x_{end} = 1
\end{aligned}
\end{equation}
Аналитическое решение выглядит следующим образом
\begin{equation}
\begin{aligned}
    y_{1} &= 2 \cos(x) - \sin(x) - e^{-x} \\
    y_{2} &= 2 \sin(x) + \cos(x)
\end{aligned}
\end{equation}
Далее приведена таблица с максимальным модулем разности с аналитическим решением и соответствующим шагом. 
\begin{center}
\begin{tabular}{ |c|c|c|c| } 
\hline
0.1& 0.05 & 0.025 & 0.0125 \\ 
$1.46050\cdot10^{-6}$ & $8.32381\cdot10^{-8}$ & $5.20788\cdot10^{-9}$ & $3.29724\cdot10^{-10}$ \\ 
\hline
\end{tabular}
\end{center}
Из чего можем сделать вывод, что порядок для этой задачи соблюдается.
Далее рассмотрим задачу
\begin{equation}
\begin{aligned}
    y_{1}' &= -y_{2} + e^{-20x} \\
    y_{2}' &= y_{1} + e^{-20x} \\
    x_{0} &= 0, y_{0} = (1,1), x_{end} = 1
\end{aligned}
\end{equation}
Аналитическое решение выглядит следующим образом
\begin{equation}
\begin{aligned}
    y_{1} &= \frac{422}{401} \cos(x) - \frac{420}{401} \sin(x) - \frac{21}{401 e^{20x}} \\
    y_{2} &= \frac{420}{401} \cos(x) + \frac{422}{401} \sin(x) - \frac{19}{401 e^{20x}}
\end{aligned}
\end{equation}
Далее приведена таблица с максимальным модулем разности с аналитическим решением и соответствующим шагом.
\begin{center}
\begin{tabular}{ |c|c|c|c| } 
\hline
0.1& 0.05 & 0.025 & 0.0125 \\ 
$2.97881\cdot10^{-4}$ & $2.10493\cdot10^{-5}$ & $1.37051\cdot10^{-6}$ & $8.69485\cdot10^{-8}$ \\ 
\hline
\end{tabular}
\end{center}
\pagebreak

\section{Устойчивость метода}
\subsection{Функция устойчивости в общем виде}
\begin{equation}
\begin{cases}
    y_{1}'(x) = \lambda y_{2}(x) \\
    y_{2}'(x) = \lambda y_{1}(x)
\end{cases}
\label{eq:test_system}
\end{equation}

Для получения функции устойчивости для предложенного метода рассмотрим результат применения одного шага метода (\ref{eq:method}) к системе (\ref{eq:test_system}). Можем рассматривать уравнение с одинаковыми собственными числами, поскольку систему с любыми двумя собственными числами можно свести к такому виду \cite{stability}.

\begin{equation}
\begin{aligned}
    y_{11} = y_{10} + z b_1^T Y_{12} \\
    y_{21} = y_{20} + z b_2^T Y_{21}
\end{aligned}    
\end{equation}

\begin{equation}
\begin{aligned}
    Y_{12i} = (1 - v_{1i}) y_{20} + v_{1i} y_{21} + \sum_{j=1}^{i-1} x_{12ij} f_{2} (Y_{21j}) \\
    Y_{21i} = (1 - v_{2i}) y_{10} + v_{2i} y_{11} + \sum_{j=1}^{i} x_{21ij} f_{1} (Y_{12j})
\end{aligned}
\end{equation}

\begin{equation}
\begin{aligned}
    Y_{21} = (I_{s2} - v_{2}) y_{10} + v_{2} y_{11} + z X_{21} Y_{12} \\
    Y_{12} = (I_{s1} - v_{1}) y_{20} + v_{1} y_{21} + z X_{12} Y_{21}
\end{aligned}
\end{equation}

\begin{equation}
    \begin{pmatrix} Y_{12}\\ Y_{21} \end{pmatrix}
    = \begin{pmatrix}
        E & -z X_{12} \\
        -z X_{21} & E
    \end{pmatrix}^{-1}
    \begin{pmatrix}
        (I_{s1} - v_{1}) y_{20} + v_{1} y_{21} \\
        (I_{s2} - v_{2}) y_{10} + v_{2} y_{11}
    \end{pmatrix}
\end{equation}

\begin{equation}
    \begin{pmatrix} y_{11}\\ y_{21} \end{pmatrix}
    = 
    \begin{pmatrix} y_{10}\\ y_{20} \end{pmatrix}
    + z
    \begin{pmatrix}
         b_{1}^{T} & 0 \dotsc 0 \\
         0 \dotsc 0 & b_{2}^{T}
    \end{pmatrix}
    \begin{pmatrix}
        E & -z X_{12} \\
        -z X_{21} & E
    \end{pmatrix}^{-1}
    \begin{pmatrix}
        (I_{s1} - v_{1}) y_{20} + v_{1} y_{21} \\
        (I_{s2} - v_{2}) y_{10} + v_{2} y_{11}
    \end{pmatrix}
\end{equation}
\begin{equation}
\begin{pmatrix}
         b_{1}^{T} & 0 \dotsc 0 \\
         0 \dotsc 0 & b_{2}^{T}
    \end{pmatrix}
    \begin{pmatrix}
        E & -z X_{12} \\
        -z X_{21} & E
    \end{pmatrix}^{-1}
=
\begin{pmatrix}
    d_{11} & d_{12} \\
    d_{21} & d_{22}
\end{pmatrix}
\end{equation}
\begin{equation}
    \begin{pmatrix} y_{11}\\ y_{21} \end{pmatrix}
    = 
    \begin{pmatrix} y_{10}\\ y_{20} \end{pmatrix}
    + z
    \begin{pmatrix}
    d_{11} & d_{12} \\
    d_{21} & d_{22}
    \end{pmatrix}
    \left[
    \begin{pmatrix}
        (I_{s1} - v_{1}) y_{20} \\
        (I_{s2} - v_{2}) y_{10} 
    \end{pmatrix}
    +
    \begin{pmatrix}
        v_{1} y_{21} \\
        v_{2} y_{11} 
    \end{pmatrix}
    \right]
\end{equation}
\begin{equation}
\begin{pmatrix}
    1 - z d_{12} v_{2} & -z d_{11} v_{1} \\
    -z d_{22} v_{2} & 1 - z d_{21} v_{1}
\end{pmatrix}
\begin{pmatrix} y_{11}\\ y_{21} \end{pmatrix}
=
\begin{pmatrix}
    1 + z d_{12} (I_{s2} - v_{2}) & z d_{11} (I_{s1} - v_{1}) \\
    z d_{22} (I_{s2} - v_{2}) & 1 + z d_{21} (I_{s1} - v_{1})
\end{pmatrix}
\begin{pmatrix} y_{10}\\ y_{20} \end{pmatrix}
\end{equation}
\begin{equation}
P(z, w_{1}, w_{2}) = 
\begin{pmatrix}
    1 + z d_{12} w_{2} & z d_{11} w_{1} \\
    z d_{22} w_{2} & 1 + z d_{21} w_{1}
\end{pmatrix}
\end{equation}
\begin{equation}
    R(z) = P^{-1}(z , -v_{1}, -v_{2}) P(z, I_{s1} - v_{1}, I_{s2} - v_{2})
    \label{eq:stability_general}
\end{equation}
\begin{equation}
\begin{pmatrix} y_{11}\\ y_{21} \end{pmatrix}
    = R(z)
\begin{pmatrix} y_{10}\\ y_{20} \end{pmatrix}
\end{equation}

\subsection{Функция устойчивости для метода}

Подставим в (\ref{eq:stability_general}) коэффициенты из \ref{conditions}.
\begin{equation}
\begin{pmatrix}
    \frac{-1 - \sqrt{2}}{2 \sqrt{2} + 5} & \frac{3}{4} &  \frac{6 \sqrt{2} + 9}{8 \sqrt{2} + 20} & 0 & 0 \\
    0 & 0 & 0 & \frac{1}{2} & \frac{1}{2}
\end{pmatrix}
\begin{pmatrix}
    1 & 0 & 0 & 0 & 0\\
    0 & 1 & 0 & -z (\frac{-\sqrt{6}}{18} + \frac{\sqrt{3}}{6}) & 0\\
    0 & 0 & 1 & -z (\frac{\sqrt{6}}{6} - \frac{\sqrt{3}}{18}) & 0\\
    z \frac{1}{6} & 0 & 0 & 1 & 0 \\
    -z (\frac{1}{6} + \frac{\sqrt{2}}{3}) & z & 0 & 0 & 1
\end{pmatrix}^{-1}
=
 \begin{pmatrix}
    d_{11} & d_{12} \\
    d_{21} & d_{22}
\end{pmatrix}
\end{equation}
\begin{equation}
\begin{gathered}
    d_{11} = \left[
    \frac{-\sqrt{3} (36 \sqrt{6} + 17 z^{2} + 12 \sqrt{3})}{612}, \frac{3}{4}, \frac{21}{68} + \frac{3 \sqrt{2}}{17}
    \right] \\
    d_{12} = \left[
    \frac{\sqrt{3} z}{6}, 0
    \right] \\
    d_{21} = \left[
    \frac{-(\sqrt{6} - 3 \sqrt{3}) z (36 \sqrt{6} + 7 z^{2} + 24 \sqrt{3})}{1512}, \frac{-z}{2}, 0
    \right] \\
    d_{22} = \left[
    \frac{(\sqrt{6} - 3 \sqrt{3}) (-18 \sqrt{3} + 7 z^{2} - 6 \sqrt{6})}{252}, \frac{1}{2}
    \right]
\end{gathered}
\end{equation}
\begin{equation}
\begin{aligned}
    R(z) = \\
\begin{pmatrix}
    \frac{(-11 + 6 \sqrt{3}) (13 z^{4} - 228 \sqrt{3} z^{2} - 132 z^{2} - 576 \sqrt{3} - 432)}{13 (z^{4} - 12 \sqrt{3} z^{2} + 12 z^{2} + 288 \sqrt{3} - 432)} & 
    \frac{(-2 + \sqrt{3}) (z^{2} + 12) (z^{2} - 12 \sqrt{3}) z}{z^{4} - 12 \sqrt{3} z^{2} + 12 z^{2} + 288 \sqrt{3} - 432}\\
    \frac{12 (-5 + 3 \sqrt{3}) (z^{2} + 6 \sqrt{3} + 18) z}{z^{4} - 12 \sqrt{3} z^{2} + 12 z^{2} + 288 \sqrt{3} - 432} &
    \frac{(-11 + 6 \sqrt{3}) (13 z^{4} - 228 \sqrt{3} z^{2} - 132 z^{2} - 576 \sqrt{3} - 432)}{13 (z^{4} - 12 \sqrt{3} z^{2} + 12 z^{2} + 288 \sqrt{3} - 432)}
\end{pmatrix}
\end{aligned}
    \label{eq:stability_function}
\end{equation}

\subsection{Облась устойчивости}
Далее для нахождения области устойчивости рассмотрим собственные числа (\ref{eq:stability_function}).
{\footnotesize
\begin{multline}
    \lambda_{1} = \frac{-(6 \sqrt{3} z^{4} - 11 z^{4} + 132 \sqrt{3} z^{2} - 204 z^{2} + 288 \sqrt{3}}{-z^{4} + 12 \sqrt{3} z^{2} - 12 z^{2} - 288 \sqrt{3} + 432} + \\ \frac{2 \sqrt{57 z^8 - 33 \sqrt{3} z^8 + 2304 z^{6} - 1332 \sqrt{3} z^{6} + 28512 z^{4} - 16416  \sqrt{3} z^{4} + 108864 z^{2} - 62208 \sqrt{3} z^{2}} - 432)}{-z^{4} + 12 \sqrt{3} z^{2} - 12 z^{2} - 288 \sqrt{3} + 432} \\
    \lambda_{2} = \frac{-(6 \sqrt{3} z^{4} - 11 z^{4} + 132 \sqrt{3} z^{2} - 204 z^{2} + 288 \sqrt{3}}{(-z^{4} + 12 \sqrt{3} z^{2} - 12 z^{2} - 288 \sqrt{3} + 432)} - \\ \frac{2 \sqrt{57 z^{8} - 33 \sqrt{3} z^{8} + 2304 z^{6} - 1332 \sqrt{3} z^{6} + 28512 z^{4} - 16416 \sqrt{3} z^{4} + 108864 z^{2} - 62208 \sqrt{3} z^{2}} - 432)}{(-z^{4} + 12 \sqrt{3} z^{2} - 12 z^{2} - 288 \sqrt{3} + 432)}
\end{multline}}
\pagebreak

\specialsection{Выводы}
Здесь выводы
\pagebreak

\specialsection{Заключение}
тут заключение
\pagebreak

\begin{thebibliography}{1}
\bibitem{code} https://github.com/potomushozhenya/qualifWork
\bibitem{srk} Олемской И. В. \flqq Методы интегрирования систем структурно разделенных дифференциальных уравнений\frqq. Издательство С.-Петербургского университета, 2009, стр. 69-91.
\bibitem{tree} Хайрер Э., Нёрсетт С., Ваннер Г. \flqq Решение обыкновенных дифференциальных уравнений\frqq. Издательство Москва Мир, 1990, стр. 150-163.
\bibitem{mirk} K. Burrage, F. H. Chipman, P. H. Muir \flqq Order results for Mono-Implicit Runge-Kutta methods\frqq.  SIAM J. Numer. Anal. Vol. 31, No. 3, pp. 876-891, 1994
\bibitem{stability} Винничек Н. Н. \flqq Численная устойчивость разделяющихся методов решения систем обыкновенных дифференциальных уравнений\frqq  Санкт-Петербургский государственный университет, 2018, стр. 9-31
\end{thebibliography}
\pagebreak

\specialsection{Приложение}
\end{document}